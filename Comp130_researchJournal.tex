\documentclass{scrartcl}

\usepackage[hidelinks]{hyperref}
\usepackage[none]{hyphenat}
\usepackage{setspace}
\doublespace

\usepackage{amsmath}

\title{Virtual Reality - Uses for, Technology behind and potential future.}

\subtitle{COMP130 - Research Journal}

\author{1605708}

\begin{document}

\maketitle

\section{Uses in Medicine}

Virtual Reality is extremely appealing for the medical industry, as it allows for simulations which can help patients get over trauma. Recently a group has created a prototype to recreate social environments in VR with specific focus on facial expressions. This is planned to be used to help children with autism learn how to better read facial expressions while they are in a safe environment\cite{lahiri2011design}. It uses eye tracking hardware to keep track of someone a patients eyes to help give further insight into what the patient is thinking.  
This use of VR is particularly interesting because it incorporates other devices into VR in order to collect data whilst also trying to help patients with the base VR simulation. While in it's current stages it is a more just used for proof of concept, there is large potential for software like this in medicine. Using eye tracking techniques in unison with VR could help with other medical and psychological issues. For example data could be collected from someone suffering from PTSD while they are in a simulated situation that could induce and help to cure their trauma. This data could then be used to create and enhance the simulations. 

\section{Uses in Marketing}

Virtual reality in marketing and businesses is interesting, mainly because access to VR is somewhat limited. VR has been used by the Lake District to give viewers a bird's eye view of the area. The paper based around this was interesting as it implies that destination managers should use VR as an opportunity to market their locations. 
This case study shows that VR is capable of more than just playing games\cite{VRtourists}, it also has the potential to become a marketing tool. By showing off a location in interesting ways through a virtual reality headset, people may be more inclined to go to destinations and see them in real life.

\section{VR in Art and Storytelling}

VR being a visually based device, means that it is natural for it to become used for storytelling (e.g series, films, videos). Whilst some VR applications simulate a theater, putting you in place of a person watching a screen a more interesting way to use VR is to put you in place of a person within the story being told. For example Office Diva\cite{Diva}, an "audio visual art installation which tells the story of a receptionist who is dealing with manic depression.  
While this is very similar to a game, where you are playing the role of the character, it tells a story more specifically whilst putting you in the same position of the character. That is to say you aren’t the character, you are viewing the situation from the characters perspective. 
This is interesting because it allows a viewer to see a situation from the same point of view as the character and gain greater insight into how they may be feeling and what they may be thinking. However it also still keeps the viewer as an external third party viewing a situation, giving them the opportunity to reflect on the characters actions and allow them to form their own opinions. 

\section{VR in Scientific Research and Simulation Software}

VR is widely used for simulation software, especially in training. VR is a powerful tool for simulations and has recently been used for scientific simulations. Many companies have taken an interest in creating "cyborg bees". To do this however a lot of research is being done into how insects fly. By obtaining data from bees in flight and translating that into a simulation using VR scientists can see how their cyborg bees will fly from the point of view of the cyborg bee\cite{zheng2017real}. Which allows for testing without damaging products. Once scientists have found a working formula and tested it a lot, they can translate it to the actual cyborg bees and test those with a much higher chance of them working correctly. 
This study shows the power of VR in scientific use. By allowing scientists to view a simulation from a new perspective and giving them the freedom that a VR headset allows insight into the simulation can be more easily obtained. This could allow scientists to develop new ideas to test and help to make new discoveries, making new contributions to the scientific community. 

\section{Future uses for VR}
 
From the previous papers we can see that there are many uses for VR and many ways we can integrate other technology into VR systems. So what else can we expect from VR. One study shows the feasibility of a VR system connecting to the internet through home routers\cite{WirelessVR}. Overall the paper showed that this was plausible theoretically, however there is no current solution to connect a VR headset directly to a router. However that does not mean it will not be so in the future.  
Whilst nothing is currently available there are many implications of this on the future of VR. For games this could lead to many more multiplayer VR titles being released, filling in a market which, while niche, is also very sparse. For a solution like this to be implemented a VR headset would need it's own operating system and hardware would have to become much lighter to allow for users to wear them on their heads. Alternatively a solution which allows VR headsets to become wireless could allow for users their VR headsets to their PC's, allowing access to online VR titles through internet connections on their PC's in the same way the play regular online games.  
Creating new hardware for VR titles is interesting as it is not something that is often done, but could help a large amount with player immersion. For example one group has created a device which can simulate changes in weight\cite{zenner2017shifty}. This can help with simulating weight of objects in virtual reality. By moving the weight within a rod the user holds, different weights can be simulated, as the user will have to apply more force based on how far away the weight is.  
VR titles suffer from some major problems when translating real life actions to in game and translating in game actions to give real life feedback. Devices like this help to create further immersion. For players this means that they can have a more enjoyable experience. For developers it opens new possibilities for game development. Additionally other hardware may be developed to solve other issues. For example issues such as the inability to control movement within titles could be solved. 

\bibliography{comp130Ref}

\end{document}